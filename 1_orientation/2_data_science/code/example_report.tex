\documentclass[]{article}
\usepackage{lmodern}
\usepackage{amssymb,amsmath}
\usepackage{ifxetex,ifluatex}
\usepackage{fixltx2e} % provides \textsubscript
\ifnum 0\ifxetex 1\fi\ifluatex 1\fi=0 % if pdftex
  \usepackage[T1]{fontenc}
  \usepackage[utf8]{inputenc}
\else % if luatex or xelatex
  \ifxetex
    \usepackage{mathspec}
  \else
    \usepackage{fontspec}
  \fi
  \defaultfontfeatures{Ligatures=TeX,Scale=MatchLowercase}
\fi
% use upquote if available, for straight quotes in verbatim environments
\IfFileExists{upquote.sty}{\usepackage{upquote}}{}
% use microtype if available
\IfFileExists{microtype.sty}{%
\usepackage{microtype}
\UseMicrotypeSet[protrusion]{basicmath} % disable protrusion for tt fonts
}{}
\usepackage[margin=1in]{geometry}
\usepackage{hyperref}
\hypersetup{unicode=true,
            pdftitle={Introduction to R Markdown},
            pdfauthor={Phil Chodrow},
            pdfborder={0 0 0},
            breaklinks=true}
\urlstyle{same}  % don't use monospace font for urls
\usepackage{color}
\usepackage{fancyvrb}
\newcommand{\VerbBar}{|}
\newcommand{\VERB}{\Verb[commandchars=\\\{\}]}
\DefineVerbatimEnvironment{Highlighting}{Verbatim}{commandchars=\\\{\}}
% Add ',fontsize=\small' for more characters per line
\usepackage{framed}
\definecolor{shadecolor}{RGB}{248,248,248}
\newenvironment{Shaded}{\begin{snugshade}}{\end{snugshade}}
\newcommand{\AlertTok}[1]{\textcolor[rgb]{0.94,0.16,0.16}{#1}}
\newcommand{\AnnotationTok}[1]{\textcolor[rgb]{0.56,0.35,0.01}{\textbf{\textit{#1}}}}
\newcommand{\AttributeTok}[1]{\textcolor[rgb]{0.77,0.63,0.00}{#1}}
\newcommand{\BaseNTok}[1]{\textcolor[rgb]{0.00,0.00,0.81}{#1}}
\newcommand{\BuiltInTok}[1]{#1}
\newcommand{\CharTok}[1]{\textcolor[rgb]{0.31,0.60,0.02}{#1}}
\newcommand{\CommentTok}[1]{\textcolor[rgb]{0.56,0.35,0.01}{\textit{#1}}}
\newcommand{\CommentVarTok}[1]{\textcolor[rgb]{0.56,0.35,0.01}{\textbf{\textit{#1}}}}
\newcommand{\ConstantTok}[1]{\textcolor[rgb]{0.00,0.00,0.00}{#1}}
\newcommand{\ControlFlowTok}[1]{\textcolor[rgb]{0.13,0.29,0.53}{\textbf{#1}}}
\newcommand{\DataTypeTok}[1]{\textcolor[rgb]{0.13,0.29,0.53}{#1}}
\newcommand{\DecValTok}[1]{\textcolor[rgb]{0.00,0.00,0.81}{#1}}
\newcommand{\DocumentationTok}[1]{\textcolor[rgb]{0.56,0.35,0.01}{\textbf{\textit{#1}}}}
\newcommand{\ErrorTok}[1]{\textcolor[rgb]{0.64,0.00,0.00}{\textbf{#1}}}
\newcommand{\ExtensionTok}[1]{#1}
\newcommand{\FloatTok}[1]{\textcolor[rgb]{0.00,0.00,0.81}{#1}}
\newcommand{\FunctionTok}[1]{\textcolor[rgb]{0.00,0.00,0.00}{#1}}
\newcommand{\ImportTok}[1]{#1}
\newcommand{\InformationTok}[1]{\textcolor[rgb]{0.56,0.35,0.01}{\textbf{\textit{#1}}}}
\newcommand{\KeywordTok}[1]{\textcolor[rgb]{0.13,0.29,0.53}{\textbf{#1}}}
\newcommand{\NormalTok}[1]{#1}
\newcommand{\OperatorTok}[1]{\textcolor[rgb]{0.81,0.36,0.00}{\textbf{#1}}}
\newcommand{\OtherTok}[1]{\textcolor[rgb]{0.56,0.35,0.01}{#1}}
\newcommand{\PreprocessorTok}[1]{\textcolor[rgb]{0.56,0.35,0.01}{\textit{#1}}}
\newcommand{\RegionMarkerTok}[1]{#1}
\newcommand{\SpecialCharTok}[1]{\textcolor[rgb]{0.00,0.00,0.00}{#1}}
\newcommand{\SpecialStringTok}[1]{\textcolor[rgb]{0.31,0.60,0.02}{#1}}
\newcommand{\StringTok}[1]{\textcolor[rgb]{0.31,0.60,0.02}{#1}}
\newcommand{\VariableTok}[1]{\textcolor[rgb]{0.00,0.00,0.00}{#1}}
\newcommand{\VerbatimStringTok}[1]{\textcolor[rgb]{0.31,0.60,0.02}{#1}}
\newcommand{\WarningTok}[1]{\textcolor[rgb]{0.56,0.35,0.01}{\textbf{\textit{#1}}}}
\usepackage{graphicx,grffile}
\makeatletter
\def\maxwidth{\ifdim\Gin@nat@width>\linewidth\linewidth\else\Gin@nat@width\fi}
\def\maxheight{\ifdim\Gin@nat@height>\textheight\textheight\else\Gin@nat@height\fi}
\makeatother
% Scale images if necessary, so that they will not overflow the page
% margins by default, and it is still possible to overwrite the defaults
% using explicit options in \includegraphics[width, height, ...]{}
\setkeys{Gin}{width=\maxwidth,height=\maxheight,keepaspectratio}
\IfFileExists{parskip.sty}{%
\usepackage{parskip}
}{% else
\setlength{\parindent}{0pt}
\setlength{\parskip}{6pt plus 2pt minus 1pt}
}
\setlength{\emergencystretch}{3em}  % prevent overfull lines
\providecommand{\tightlist}{%
  \setlength{\itemsep}{0pt}\setlength{\parskip}{0pt}}
\setcounter{secnumdepth}{0}
% Redefines (sub)paragraphs to behave more like sections
\ifx\paragraph\undefined\else
\let\oldparagraph\paragraph
\renewcommand{\paragraph}[1]{\oldparagraph{#1}\mbox{}}
\fi
\ifx\subparagraph\undefined\else
\let\oldsubparagraph\subparagraph
\renewcommand{\subparagraph}[1]{\oldsubparagraph{#1}\mbox{}}
\fi

%%% Use protect on footnotes to avoid problems with footnotes in titles
\let\rmarkdownfootnote\footnote%
\def\footnote{\protect\rmarkdownfootnote}

%%% Change title format to be more compact
\usepackage{titling}

% Create subtitle command for use in maketitle
\providecommand{\subtitle}[1]{
  \posttitle{
    \begin{center}\large#1\end{center}
    }
}

\setlength{\droptitle}{-2em}

  \title{Introduction to R Markdown}
    \pretitle{\vspace{\droptitle}\centering\huge}
  \posttitle{\par}
    \author{Phil Chodrow}
    \preauthor{\centering\large\emph}
  \postauthor{\par}
      \predate{\centering\large\emph}
  \postdate{\par}
    \date{Tuesday, August 27th, 2019}


\begin{document}
\maketitle

{
\setcounter{tocdepth}{2}
\tableofcontents
}
\hypertarget{literate-programming}{%
\section{Literate Programming}\label{literate-programming}}

When doing analysis, we typically generate code, plots, explanatory
text, narrative text, and other artifacts. The idea of \textbf{literate
programming} is simple:

\begin{quote}
Put analysis code, outputs, and text the \emph{same document}, and then
get different \emph{views} of that document according to the appropriate
audience.
\end{quote}

\hypertarget{markdown}{%
\section{Markdown}\label{markdown}}

Markdown is a simple language invented by
\href{https://daringfireball.net/projects/markdown/}{John Gruber}.
Originally, the purpose of Markdown was to easily generate webpages and
blog posts without the need for writing complex HTML. Markdown's primary
strength is the display of simple text, but there are lots of options:

\begin{itemize}
\tightlist
\item
  \emph{Italic} and \textbf{bold} typfaces.
\item
  Section headers like the one above this list, delimited using \#.
\item
  Especially important for our purposes, the inclusion of images (such
  as statistical graphics), as well as advanced HTML objects
  (interactive maps, for example).
\item
  Code display
\end{itemize}

\begin{Shaded}
\begin{Highlighting}[]
\KeywordTok{print}\NormalTok{(}\StringTok{"Hello World!"}\NormalTok{)}
\end{Highlighting}
\end{Shaded}

To write in Markdown, open any text editor (RStudio works great!) and
start writing! There are lots of great Markdown references online.

\hypertarget{r-markdown}{%
\section{R Markdown}\label{r-markdown}}

Markdown itself is extremely useful, but R Markdown is something
especially magical. R Markdown implements the literate programming
paradigm by enabling you to smoothly intersperse code, outputs like
tables and graphics, and expository text all in the same document. Let's
run some simple examples. First, we'll make a code chunk and load the
data set.

\begin{Shaded}
\begin{Highlighting}[]
\KeywordTok{library}\NormalTok{(tidyverse)}
\KeywordTok{library}\NormalTok{(knitr)}
\NormalTok{df <-}\StringTok{ }\KeywordTok{read_csv}\NormalTok{(}\StringTok{'../../data/listings.csv'}\NormalTok{)}
\end{Highlighting}
\end{Shaded}

\hypertarget{peeking-at-data}{%
\subsection{Peeking at Data}\label{peeking-at-data}}

For a start, we can embed views of our data. Let's look at the 10
lowest-rated listings.

\begin{Shaded}
\begin{Highlighting}[]
\NormalTok{df }\OperatorTok\StringTok{ }
\StringTok{  }\KeywordTok{select}\NormalTok{(name, neighbourhood, review_scores_rating) }\OperatorTok\StringTok{ }
\StringTok{  }\KeywordTok{arrange}\NormalTok{(review_scores_rating) }\OperatorTok\StringTok{ }
\StringTok{  }\KeywordTok{head}\NormalTok{(}\DecValTok{10}\NormalTok{) }
\end{Highlighting}
\end{Shaded}

\begin{verbatim}
## # A tibble: 10 x 3
##    name                                  neighbourhood   review_scores_rat~
##    <chr>                                 <chr>                        <dbl>
##  1 near BC shuttle house summer sublet   Allston-Bright~                 20
##  2 High-End 2-BR Furnished Apartment *F~ Fenway/Kenmore                  20
##  3 Beautifully Furnished 1-BR in Boston~ West End                        20
##  4 BOSTON Private Room/ #2 Modern/Cozy ~ Mattapan                        20
##  5 Lux 2BR 2BA Apt w/rooftop deck, gym,~ Mission Hill                    20
##  6 Lux 2BR 2BA Apt w/rooftop deck, gym,~ Mission Hill                    20
##  7 New Heart of The North End One BR Du~ North End                       20
##  8 Sweetheart of The North End 2 Bedroo~ North End                       20
##  9 Room for Rent in Allston near Green ~ Allston-Bright~                 20
## 10 Boston/Fort Hill - Bedroom in Beauti~ Roxbury                         20
\end{verbatim}

This is the magic of R Markdown: anything you can do in \texttt{R}, you
can execute, visualize, and explain in the same document.

\hypertarget{inline-computations}{%
\subsection{Inline computations}\label{inline-computations}}

As you are crafting expository text, you might wish to reference a given
number or name from the code. For example, when describing the data set,
you might run \texttt{nrow(df)} and \texttt{ncol(df)} and then manually
write in the answers:

\begin{quote}
The data has 6264 rows and 106 columns.
\end{quote}

Of course, that's correct all on its own, but\ldots{}what happens when
you get an updated data set from your collaborators? (This actually
happened to me: we updated the data set this year.) Think you're going
to remember to correct the manually-typed number in all your documents?
Try this instead:

\begin{quote}
The data set has \texttt{6264} rows and \texttt{106} columns.
\end{quote}

RMarkdown gives you the ability to neatly insert the results of your
computation into your document. This can get very handy when you want to
include slightly more complex descriptions:

\begin{Shaded}
\begin{Highlighting}[]
\NormalTok{top_rated_neighborhood <-}\StringTok{ }\NormalTok{df }\OperatorTok\StringTok{ }
\StringTok{  }\KeywordTok{group_by}\NormalTok{(neighbourhood) }\OperatorTok\StringTok{ }
\StringTok{  }\KeywordTok{summarise}\NormalTok{(}\DataTypeTok{review_scores_rating =} \KeywordTok{mean}\NormalTok{(review_scores_rating, }\DataTypeTok{na.rm =}\NormalTok{ T)) }\OperatorTok\StringTok{ }
\StringTok{  }\KeywordTok{arrange}\NormalTok{(}\KeywordTok{desc}\NormalTok{(review_scores_rating))}
\end{Highlighting}
\end{Shaded}

There are \texttt{30} neighborhoods represented in the data set. Of
these, the highest rated is \texttt{Chestnut\ Hill}. This analysis is
current as of \texttt{2019-08-27}

\hypertarget{graphics}{%
\subsection{Graphics}\label{graphics}}

Of course, the main event here is graphics. Any graphics you can
generate in \texttt{R} go seamlessly into your reporting.

\begin{Shaded}
\begin{Highlighting}[]
\NormalTok{top_rated_neighborhood }\OperatorTok\StringTok{ }
\StringTok{  }\KeywordTok{mutate}\NormalTok{(}\DataTypeTok{neighbourhood =} \KeywordTok{factor}\NormalTok{(neighbourhood, }\DataTypeTok{levels =}\NormalTok{ .}\OperatorTok{$}\NormalTok{neighbourhood)) }\OperatorTok\StringTok{ }
\StringTok{  }\KeywordTok{ggplot}\NormalTok{() }\OperatorTok{+}\StringTok{ }
\StringTok{  }\KeywordTok{aes}\NormalTok{(}\DataTypeTok{x =}\NormalTok{ neighbourhood, }\DataTypeTok{y =}\NormalTok{ review_scores_rating) }\OperatorTok{+}\StringTok{ }
\StringTok{  }\KeywordTok{geom_bar}\NormalTok{(}\DataTypeTok{stat =} \StringTok{'identity'}\NormalTok{) }\OperatorTok{+}\StringTok{ }
\StringTok{  }\KeywordTok{coord_flip}\NormalTok{() }\OperatorTok{+}\StringTok{ }
\StringTok{  }\KeywordTok{theme_bw}\NormalTok{() }
\end{Highlighting}
\end{Shaded}

\begin{verbatim}
## Warning: Removed 1 rows containing missing values (position_stack).
\end{verbatim}

\includegraphics{example_report_files/figure-latex/unnamed-chunk-5-1.pdf}

\hypertarget{math-and-code}{%
\subsection{Math and Code}\label{math-and-code}}

RMarkdown is well-integrated with MathJax, which allows you to write
\(\LaTeX\)-style mathematics within your document.

\[f(\beta) \triangleq \lVert \mathbf{y} - \mathbf{X}\beta \rVert_2^2\]

You can display and even execute code -- \texttt{R} is included, of
course, but how about this:

\begin{Shaded}
\begin{Highlighting}[]
\ImportTok{import}\NormalTok{ sys}
\BuiltInTok{print}\NormalTok{(}\StringTok{"This is Python "} \OperatorTok{+}\NormalTok{ sys.version)}
\end{Highlighting}
\end{Shaded}

\begin{verbatim}
## This is Python 2.7.10 (default, Feb 22 2019, 21:55:15) 
## [GCC 4.2.1 Compatible Apple LLVM 10.0.1 (clang-1001.0.37.14)]
\end{verbatim}

\hypertarget{resources}{%
\section{Resources}\label{resources}}

\begin{itemize}
\tightlist
\item
  The \href{https://rmarkdown.rstudio.com/}{R Markdown website},
  including documentation for reports, slides, dashboards, and more.
\item
  \href{https://rmarkdown.rstudio.com/flexdashboard/index.html}{Flexdashboard
  reference} for working on your project presentations.
\item
  \href{https://en.wikipedia.org/wiki/Literate_programming}{Introduction
  to literate programming}
\end{itemize}


\end{document}
